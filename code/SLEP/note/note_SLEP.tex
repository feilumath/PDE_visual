
\documentclass[12pt]{article}
%%%%%%%%%%%%%%%%%%%%%%%%%%%%%%%%%%%%%%%%%%%%%%%%%%%%%%%%%%%%%%%%%%%%%%%%%%%%%%%%%%%%%%%%%%%%%%%%%%%%%%%%%%%%%%%%%%%%%%%%%%%   packages    %%%%%%%%%
\usepackage[small,compact]{titlesec}
\usepackage{amsmath,amssymb,amsfonts,mathabx,setspace}
\usepackage{graphicx,caption,epsfig,subfigure,epsfig,wrapfig}
\usepackage{url,color,verbatim,algorithmic}
\usepackage[sort,compress]{cite}
\usepackage[ruled,boxed]{algorithm}
\usepackage[scaled]{helvet}
\usepackage[T1]{fontenc}
\usepackage[bookmarks=true, bookmarksnumbered=true, colorlinks=true,   pdfstartview=FitV,
linkcolor=blue, citecolor=blue, urlcolor=blue]{hyperref}


%%%%%%%%%%%----   page settings   ----- %%%%%%%%%
\marginparwidth 0pt
\oddsidemargin  -0.15in   % = 0.38 cm
\evensidemargin  -0.15in % = 0.38 cm
\marginparsep 0pt
\topmargin   -.55in   % = 1.40 cm
\textwidth   6.8in      % = 17.27 cm
\textheight  9.3in      % = 23.87 cm
%\footskip 3mm
%\setstretch{1.05}  \hoffset=-0.1in  \voffset= -0.8in
\parskip=0.0in


%%%%%%%%%%%----  new commands/defs   ----- %%%%%%%%%
\def\clattice{c_{\mathrm{lat}}}           
\def\Rn{\mathbb{R}^n}
\def\R{\mathbb{R}}                            
\def\P{\mathbb{P}}
\def\bbeta{\bm \beta}                        
\newcommand{\E}[1]{\mathbb{E}\left[{#1}\right]}
\newcommand{\rbracket}[1]{\left(#1\right)}
\newcommand{\sbracket}[1]{\left[#1\right]}
\newcommand{\norm}[1]{\left\|#1\right\|}
\newcommand{\normv}[1]{\left| #1\right|}
\newcommand{\innerp}[1]{\langle{#1}\rangle}
\newcommand{\dbinnerp}[1]{\langle\langle{#1}\rangle\rangle}
\newcommand{\vect}[1] {\pmb{#1}}
\newcommand{\mat}[1]{\pmb{#1}}
\newcommand{\floor}[1]{\lfloor{#1}\rfloor}

\newcommand{\Ito}{It\^o}

%% Note commands
% \usepackage{soul,array}
%% margin notes
%\newcommand{\note}[1]{\marginpar{\renewcommand{\baselinestretch}{1.0}{\scriptsize{\parbox{0.5in}{\raggedright{\bf{\sffamily #1}}}}}}}
%\newcommand{\fnote}[1]{\note{Fei: {#1}}}
%\newcommand{\frem}[1]{{\textcolor{cyan}{[Fei: {#1}]}}}
%\newcommand{\fnew}[1]{{\textcolor{blue}{#1}}}
%\newcommand{\fpar}[1]{\bigskip\noindent{\em {#1}}}
% \newcommand{\floor}[1]{{\lfloor{#1}\rfloor}}
\newcommand{\blue}[1]{\textcolor{blue}{{#1}}}
\newcommand{\frem}[1]{{\textcolor{cyan}{[Remark: {#1}]}}}
\newcommand{\fqes}[1]{{\textcolor{red}{[{\bf Question:} {#1}]}}}

%%%%%%%%%%%---- new environments  
\newtheorem{theorem}{Theorem}
\newtheorem{acknowledgement}[theorem]{Acknowledgement}
% \newtheorem{algorithm}[theorem]{Algorithm}
\newtheorem{claim}[theorem]{Claim}
\newtheorem{condition}[theorem]{Condition}
\newtheorem{conjecture}[theorem]{Conjecture}
\newtheorem{corollary}[theorem]{Corollary}
\newtheorem{definition}[theorem]{Definition}
\newtheorem{example}[theorem]{Example}
\newtheorem{lemma}[theorem]{Lemma}
\newtheorem{proposition}[theorem]{Proposition}
\newtheorem{remark}[theorem]{Remark}
\newenvironment{proof}[1][Proof]{\noindent\textbf{#1.} }{\ \rule{0.5em}{0.5em}}
% \import{/Users/feilu/Documents/ref_abc}


\begin{document}

%%% ============== Title  =======
\begin{center}
\textbf{\Large Sturm-Liouville Eigenvalue Problem and numerical solution} \\[0pt]
\vspace{4mm} Fei Lu\\
 feilu@math.jhu.edu \\
Last updated: \today  
\end{center}
 
% \title{Reduction using local average and rescaling}
% \author{}  \maketitle

\begin{abstract}
Summary: numerical solution to the SLEP by finite difference. 
 \end{abstract}


% \tableofcontents

\section{The Sturm-Liouville Eigenvalue problem}


\begin{align*}
&(p(x)\phi')'  + q(x)\phi = -\lambda \sigma \phi \\
 &  \beta_1\phi(a) + \beta_2\phi'(a) = 0;   \\ % \partial_t u(x,0) = g(x) \\
 &   \beta_3\phi(b) + \beta_4\phi'(b) = 0;   
\end{align*}


\textbf{Regular SLEP:} \vspace{-3mm}
\begin{align*}
& p', q,\sigma  \in C[a,b], \quad\\
& p(x)>0, \sigma(x)>0, \forall x\in [a,b] \\
& \beta_1^2+\beta_2^2 >0, \beta_3^2+\beta_4^2 >0, 
\end{align*}

\begin{theorem}[ Sturm-Liouville Theorems] A regular SLEP has eigenvalues and eigenfunctions $\{(\lambda_n,\phi_n)\}$ s.t.~
\begin{enumerate}
\item[1-2] $\{\lambda_n\}_{n=1}^\infty$ are real and strictly increasing to $\infty$
\item[3] $\phi_n$ is the unique (up to a multiplicative factor) solution to $\lambda_n$; 
        $\phi_n$ has $n-1$ zeros
\item[4] $\{\phi_n\}_{n=1}^\infty$ is complete. That is, any piecewise smooth $f$ can be represented by a generalized Fourier series  
$ f(x) \sim \sum_{n=1}^\infty a_n \phi_n(x) =\frac{1}{2}[f(x_-)+f(x_+)]$  
\item[5] $\{\phi_n\}_{n=1}^\infty$  are orthogonal: $\innerp{\phi_n,\phi_m}_\sigma= 0$ if $n\neq m$; $\innerp{\phi_n,\phi_n}_\sigma>0$ \
\item[6] Rayleigh quotient   $\lambda_n= -\frac{\innerp{L\phi_n,\phi_n}}{\innerp{\phi_n,\phi_n}_\sigma }$;  \quad 
\end{enumerate}
where $ \innerp{f,g} := \int_a^bf(x)g(x) dx $; \quad$ \innerp{f,g}_\sigma := \int_a^bf(x)g(x)  \sigma(x) dx$.
\end{theorem}


\bigskip
Example: when $p(x)=1$ for all $x$ and $q=0$. The eigenvalues and eigenfunctions are $\lambda_n = (\frac{n\pi }{b-a})^2$
\section{Finite difference approximations}
The simplest approximation is finite difference approximation. We approximate the 1st and 2nd order derivatives by central difference: 
\begin{align*}
y'(x) & = \frac{y(x+h) - y(x-h)}{2h} + O(h^2), \\
y''(x) & = \frac{y(x+h) + y(x-h) - 2y(x)}{h^2} + O(h^2),
\end{align*}
where the order of error follows from Taylor expansion, assuming that the function $y$ has bounded 3rd order derivatives on $[a,b]$. 

Let 


% \bibliographystyle{alpha}
%\bibliography{references}

\begin{thebibliography}{9}
\bibitem{einstein} 
Albert Einstein. 
\textit{Zur Elektrodynamik bewegter K{\"o}rper}. (German) 
[\textit{On the electrodynamics of moving bodies}]. 
Annalen der Physik, 322(10):891–921, 1905.

\bibitem{knuthwebsite} 
Knuth: Computers and Typesetting,
\\\texttt{http://www-cs-faculty.stanford.edu/\~{}uno/abcde.html}
\end{thebibliography}




\end{document}
